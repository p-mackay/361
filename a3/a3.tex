\documentclass{article}
\usepackage{enumerate}
\usepackage{titling}
\usepackage{listings}
\usepackage{changepage}
\usepackage[margin=0.75in]{geometry}

\lstset{
    basicstyle=\ttfamily,
    breaklines=true,
    columns=fullflexible
}

\setlength{\droptitle}{-2cm}
\title{CSC361: Assignment 3 (a3)}
\author{Paul MacKay}

\begin{document}

\maketitle


\section{}
Given that the server is non-blocking and uses select() 
the server does not wait (block) until an operation is complete rather it continues 
to process requests even if one request is hanging. 

\section{}
\subsection{}
Using the select() function. When a client connects to the server, a socket is opened,
and appended to the client\_socket list. The server checks the entries in this list 
to see if there are any requests from these sockets.  
\subsection{}
If no data has been requested through the socket within 30-seconds then the socket is closed.

\section{}
Using the select() function even if the requests are sent back to back, the client\_sockets 
list can be checked for any new requests.

\end{document}
